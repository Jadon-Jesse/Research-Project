\section{Introduction}
Multiple logistic challenges faced in industry can be classified as combinatorial optimization problems. The Bin Packing Problem (BPP) and the Traveling Salesman Problem (TSP) are two particular problems that posses characteristics which are applicable to various instances occurring in industries such as: the medical, engineering, manufacturing, financial, transportation and agricultural industry (amongst others). A few specific examples of problems of the BPP and TSP nature are: cargo and commuter transport scheduling, stock cutting problems, machine scheduling, packing problems and the planning of tour routes. In primitive (including animal) societies, finding the shortest tracks when searching for food and optimizing storage space used to store food for hibernation are two scenarios that portray the nature of the BPP and the TSP.  Finding solutions to combinatorial optimization problems are most pertinent to the population of the  twenty-first century; as a result of the world being highly interconnected. In fierce global markets, which comprise of rival businesses entwined with the limited resource bearing environment, it is crucial to implement measures which result in minimal cost and have the slightest impact on the environment, while generating high profits to remain competent to global economies.\\
Due to the frequent occurrence of these problems, there has been extensive research conducted in attempts to find good solutions to the BPP and the TSP, which are both considered to be NP-complete [\ref{BPP_hist2}]. NP-complete means, that all known exact solution techniques that can be applied to these problems require exponentially increasing number of iterations as the problem size grows.  Since it is not feasible to implement exact or deterministic techniques, the problems have also been approached using approximation algorithms. Approximation algorithms are efficient at computing sub-optimal solutions with a provable approximation guarantee. The algorithm usually entails relaxing some of the problem's constraints that the problem has to finding the optimal solution. These algorithm computes a feasible solution that is close to optimal.Consequently, many researchers have considered combinatorial optimization problems, which is the study of the best selection and configuration of a collection of objects adhering to some objective function [\ref{SA_5}]. Research on these topics have been conducted with aim of finding optimal results or sufficiently good solutions to these combinatorial optimization problems. In the case where computation time is prioritized over optimality, completeness and efficiency;  heuristic techniques are employed to quickly obtain an approximate results which can usually be used to make further decisions on how to approach the problem. On the other hand, we have metaheurastic techniques, which are high level approaches that guide various other heuristic techniques in search for solutions in a broad set of problem domains. A few alterations are made to well-known metaheurastic techniques, with the intention to increase their performance across a wide range of similar problems [\ref{hBPP_1}]. \\
The history of the BPP and TSP are both problems that have been highly researched by various mathematicians. Some of the very first documented research on the Stock-Cutting problem, which has many similarities to the BPP, can be dated back to 1973 [\ref{CSP_hist1}]. The initial approach to these problems were ad-hoc techniques. Recent advances in packing problems[\ref{BBP_recentAd}],use various metaheurastic and heuristic techniques, alongside greedy approaches [\ref{hBPP_1}]. The TSP dates back to as early as 1856,to when the notion of Hamiltonian circuits were discussed. The Isocosic game, designed by William Hamiltonian in 1858, is another instance of  where the characteristics of the TSP are visible [\ref{TSP_hist2}]. From as early as 1954, the TSP was approached on a larger scale, as seen in [\ref{TSP_hist3}]. In more recent time, a paper of 2015, published results of various techniques implemented to solve the TSP, including running algorithms using multi node clusters and a sequence of heuristic and metaheurastic techniques. 
\begin{flushleft}
A typical operational research scenario, occurring in some commodity groups, are one which begins from raw material sites, production of products to the transportation of goods to storage locations, as well as, from delivery locations to customers. A problem of this nature will be studied for my honours research project. Section 2 contains a formal description of the problem.  Section 3 contains the objectives of the research conducted. Preliminary research is documented under Section 4.  A proposed manner in which to obtain the set objectives, is stated in Section 5. Finally, a proposed time line for the project is given in Section 6.
\end{flushleft}
 