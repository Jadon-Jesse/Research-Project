\newpage
\section{Problem Statement}
A specific challenge faced by the market today, is having to deliver purchased merchandise from depot ports to the locations of customers. A scenario that we consider is of a company, who has a list of ordered items that are ready to be delivered. The problem can be split into two cases: 
\begin{itemize}
\item Case I: The company owns a limited variety of delivery vehicles. A selection of available delivery vehicles have to be packed with the items that are to be delivered for that day.
\item Case II: A minimum length delivery route must be set out, for travel. 
\end{itemize}
These two cases are considered in order to minimize space wastage in delivery vehicles, minimize traveling distance hence, reducing both cost incurred and the carbon-footprint created by the delivery vehicle. Conceptually, the problem at hand seems simple, however, it is much more complex. The first stage of the problem – Case I, can be classified as a BPP. The second stage of the problem – Case II, can be classified as a TSP. The problem's complexity comes into play when we consider the constraints that are imposed. Noting that the items to be delivered are not all the same in size, we have a restriction on which vehicle we consider to use. We consider the question, is it better to distribute the items among various vehicles on delivery - prioritizing delivery time, or packing the items in the minimum number of vehicles such that we are dispatching the minimum number of delivery vehicles? We will consider the latter case, where the total distance traveled on the delivery route is minimized. 

\subsection{Case I: The BPP}
The BPP component entails packing the $n$ deliverable items into the minimum number of bins without exceeding its fixed capacity and has the minimum wastage of space [\ref{hBPP_1}] . In this case our bins are the subset of available delivery vehicles, into which the items are to be packed. Note, not all the items that are required to be distributed are the same in size. In order to simplify the problem, we initially consider the problem to be one dimensional. Only one dimension of the item is not fixed, for example length, whilst the other two dimensions (width and height) remain constant. To further simplify the problem, the items can be classified as either small, medium or large – restricting the lengths to one of three options. The company has a range of delivery vehicles that it utilities to deliver the purchased merchandise. For example, the company's fleet of vehicles is composes of: a mo-pad, a delivery van and a large truck. Each vehicle has it's own carrying capacity. The constraint of the vehicle's carrying capacity is important to consider when allotting item's to vehicle's for delivery. Various strategies will be considered in order to find the optimal solution to the problem, of finding a combination of the minimum number of vehicles to be used for distribution of the $n$ deliverable items. The next step of the problem, is to find feasible solutions in two dimensions, as well as, in three dimensions.  

\subsection{Case II: The TSP}
The TSP describes a salesman wanting to visit $m$ distinct cities and then returning to his home city. The route considered must be such that the overall distance is minimized, while visiting each city not more than once [\ref{TSP_Ali6}]. The TSP component of the problem we considering, focuses on the delivery route taken by the dispatched delivery vehicle. Once Case I is complete, a list of the corresponding locations of the items packed in each of the delivery vehicles used are considered. For each of these delivery vehicles the TSP is considered, where the distinct cities are the customers locations. 

The primary aim of the combinatorial optimization problem being considered,is to find solutions to the BPP and TSP which are feasible and are calculated within some reasonable time – with a polynomial performance complexity. 