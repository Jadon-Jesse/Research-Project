\newpage
\section{Objectives}
The objective of our research is to solve the distribution problem of delivering purchased merchandise from depot ports to the locations of customers using combinatorial solution techniques. Specifically, considering techniques which can be applied to the BPP and the TSP, such that we can obtain solutions which are near optimal, if not optimal, within a reasonable amount of time (with a polynomial performance complexity). Using various advantageous elements of the techniques studied, we anticipate implementation of these combined techniques to solve the problem, composing of the BPP and the TSP, which can be used in various industries as mentioned in the section 1. 

Specifically, for the BPP, we seek to minimize the number of delivery vehicles (bins) that will be unstilted for the delivery of purchased goods without exceeding the capacity of the vehicle (bin). Furthermore, the solution we seek should have minimum wastage of space and should allot item's to appropriate vehicles (bins). The initial solution we seek will be for a one dimensional problem. We then will consider a two dimensional problem. The implementation of this solution will be to read in a list of vehicles that are available with it's associated carrying capacity, as well as, a list of items which are to be distributed with it's dimensions. The algorithm devised should use the information to allocate the items to the appropriate vehicles. The algorithm should then output the list of items to be allocated to a specific vehicle such that the number of vehicles are minimized and that the execution time is reasonable.  If time permits, an implementation for a three dimensional problem will be investigated. 

Once the item's have been allocated to a subset of the vehicles, we move to the second stage of the problem – the TSP component. Each of the delivery vehicles that have been packed with items are considered. The items in a delivery vehicle have a corresponding drop off location. A list of these locations are used to design a delivery route for the specific delivery vehicle. The delivery route should visit each of the drop off locations with the minimum travel distance.

With the objective of finding the best solution to the problem, we consider exhaustive parameters per each day delivery. We consider situations where delivery time may not be suitable for the customer or if the customer cancels their order once the delivery vehicle has already begun his tour, thus, requiring rescheduling or rerouting to be performed. We aim to implement a technique such that the recalculation of the route is optimized, without having to visit the locations which no longer are to be visited. 

